\documentclass{jlreq}

\usepackage{listings}
\usepackage{caption}
\usepackage{fancyhdr}
\usepackage{graphicx}
\usepackage{textcomp}
\usepackage{multirow}
\usepackage{here}
\usepackage{amssymb,amsmath,latexsym,mathtools}

% \lstset{
%     language=R, % 使用するプログラム言語を指定
%     basicstyle=\ttfamily\footnotesize, % フォントの指定
%     numbers=left, % 行番号を表示(必要な場合)
%     numberstyle=\tiny, % 行番号のスタイル
%     frame=single, % ソースコードを枠で囲む(必要な場合)
%     breaklines=true, % 長い行を自動的に折り返す
%     captionpos=b, % キャプションの位置を下にする
% }
% \renewcommand{\lstlistingname}{ソースコード}

\pagestyle{fancy}
\fancyhf{} % 既存のヘッダーとフッターをクリア
\fancyhead[R]{\thepage}% 右上にページ番号を配置
\setlength{\headheight}{17.0pt}
\addtolength{\topmargin}{-7.0pt}

\begin{document}

\tableofcontents
\clearpage

\section{実験目的}
ミュラーリヤー錯視図形を例として,刺激条件と認知との間の法則性を理解するとともに,認知特性の測定方法や分析方法の一端を学ぶ.

\section{実験方法}
配布された標準刺激の折り目を内側に曲げて比較刺激を差し込み,上昇系列と下降系列に合わせて比較刺激をスライドさせ,標準刺激の主線と等しい長さに見える位置(PSE)まで調整した.
その後,標準刺激を差し替えて5種類の条件を順次変化させた.この際,順序効果による課題への影響を避けるために,被験者によって課題の順番を変えた.
また,上昇系列と下降系列の2条件とも4回ずつ計8回測定し,8つのPSEを求め,それを5種類の標準刺激それぞれについて行った.
5つの刺激条件それぞれについて,上昇系列(A)と下降系列(D)を別々にしてPSEの平均値,錯視量($I$)を求めた.求まった2班分(15-16名)のデータを統計分析に用いるために一つのファイルに集め,それらのデータをもとに統計分析を行った.

\section{実験結果}
同じ矢羽の角度であっても,上昇系列(A)と下降系列(D)で錯視量($I$)に相違があるのかを明らかにするために,表\ref{tab:myPSE}と表\ref{tab:groupIllusion}にデータをまとめた.
ここで,錯視量($I$)の式は以下のように求めた.
\begin{equation}
    I = 10 - \text{PSE}
\end{equation}
表\ref{tab:myPSE}は著者のデータであり,表\ref{tab:groupIllusion}は著者のデータに加えて班員のデータをまとめたものである.
そして,これらのデータについて,5つの刺激条件毎に,錯視量に相違があるかをt検定により調べると表のようになった.

\begin{table}[H]
    \centering
    \caption{レポート著者によるPSE,その合計と平均値,錯視量($I$)}
    \label{tab:myPSE}
    \begin{tabular}{|r|l|r|r|r|r|r|}
        \hline
        施行順 & 条件 & 240\textdegree & 60\textdegree & 300\textdegree& 180\textdegree & 120\textdegree \\ \hline
        1 & A & 11 & 8.6 & 11.5 & 9.5 & 8.8 \\ \hline
        2 & D & 8.4 & 10.7 & 7.7 & 9.4 & 9.5 \\ \hline
        3 & D & 8.3 & 10.6 & 7.7 & 9.8 & 9.6 \\ \hline
        4 & A & 10.9 & 8.4 & 11.3 & 9.5 & 8.8 \\ \hline
        5 & A & 10.7 & 9.1 & 11.4 & 9.9 & 8.6 \\ \hline
        6 & D & 8.1 & 9.7 & 7.4 & 9.7 & 9.6 \\ \hline
        7 & D & 8.5 & 10.5 & 7 & 9.7 & 9.5 \\ \hline
        8 & A & 10.8 & 8.8 & 11.6 & 10 & 8.6 \\ \hline 
        \multicolumn{1}{c}{} & \multicolumn{1}{c}{} & \multicolumn{1}{c}{} & \multicolumn{1}{c}{} & \multicolumn{1}{c}{} & \multicolumn{1}{c}{} & \multicolumn{1}{c}{} \\ \hline
        \multirow{3}{*}{条件A}
            & 合計 & 43.4 & 34.9 & 45.8 & 38.9 & 34.8 \\ \cline{2-7}
            & 平均 & 10.85 & 8.725 & 11.45 & 9.725 & 8.7 \\ \cline{2-7}
            & 錯視量($I$) & -0.85 & 1.275 & -1.45 & 0.275 & 1.3 \\ \hline
        \multirow{3}{*}{条件D}
            & 合計 & 33.3 & 41.5 & 29.8 & 38.6 & 38.2 \\ \cline{2-7}
            & 平均 & 8.325 & 10.375 & 7.45 & 9.65 & 9.55 \\ \cline{2-7}
            & 錯視量($I$) & 1.675 & -0.375 & 2.55 & 0.35 & 0.45 \\ \hline
    \end{tabular}
\end{table}

\begin{table}[H]
    \centering
    \caption{2班分の条件と角度ごとの錯視量($I$)のまとめ}
    \label{tab:groupIllusion}
    \begin{tabular}{|c|r|r|r|r|r|r|r|r|r|r|}
        \hline
        sub.NO & 60A & 60D & 120A & 120D & 180A & 180D & 240A & 240D & 300A & 300D \\ \hline
        1 & 0.825 & -0.1 & 1.325 & 0.675 & 0.6 & 0.85 & -0.75 & -1.1 & -1.2 & -1.65 \\ \hline
        2 & 1.1 & 0.75 & 0.8 & 0.625 & 0.225 & 0.425 & -0.275 & 0.525 & -1.375 & -0.925 \\ \hline
        3 & 2.575 & 1.625 & 2.25 & 1.3 & 1.15 & 0.725 & -0.025 & -0.225 & -0.125 & -0.575 \\ \hline
        4 & 0.58 & 0.485 & 0.2275 & 0.015 & -0.0325	& -0.0325 & -1.9375	& -1.8725 & -1.8575	& -1.795 \\ \hline
        5 & 0.9	& 1.325	& 0.5 & 0.8	& 0.575	& 0.025	& -0.325 & -0.7	& -0.65	& -0.2 \\ \hline
        6 & 0.75 & 0.775 & 1.025 & 0.425 & 0.05	& 0.125	& -0.5 & -0.925	& -1 & -1.25 \\ \hline
        7 & 1.3	& 1.4 & 0.725 & 1.4	& 0	& 0.15 & -0.725	& -0.525 & -0.925 & -0.9 \\ \hline
        8 & 1.275 & -0.375 & 1.3 & 0.45	& 0.275	& 0.35 & -0.85 & 1.675 & -1.45 & 2.55 \\ \hline
        9 & 1.28 & 0.55	& 0.68 & 0.7 & -0.05 & -0.1	& -0.38	& -0.35	& -1.35	& -0.83 \\ \hline
        10 & 1.4 & 1.8 & 0.7 & 0.85	& 0.025	& 0.075	& -0.275 & 0.325 & -0.35 & -0.425 \\ \hline
        11 & 1.75 & 1.925 & 0.725 & 1.075 & 0.15 & -0.075 & 0.075 & -0.075 & -0.725	& -0.85 \\ \hline
        12 & 1.425 & 0.275 & 1.125 & -0.175	& 0.9 & 0.075 & -0.375 & -1.3 & -1.25 & -1.625 \\ \hline
        13 & 0.4 & -0.275 & 0.7	& 0.125	& 0.275	& 0.05 & -0.125	& -0.55	& -0.4 & -0.725 \\ \hline
        14 & -0.6 & -0.45 & 0.65 & -0.6	& -0.25	& -0.45	& -1.075 & -1.325 & -1.85 & -1.35 \\ \hline
        15 & 1.325 & 1.25 & 1.95 & 1.1 & 0.4 & 0.45 & 0.325	& -0.425 & -0.425 & -0.75 \\ \hline
    \end{tabular}
\end{table}

\section{考察}

\end{document}