\documentclass{jlreq}

\usepackage{listings}
\usepackage{caption}
\usepackage{fancyhdr}
\usepackage{graphicx}

\lstset{
    language=C, % 使用するプログラム言語を指定
    basicstyle=\ttfamily\footnotesize, % フォントの指定
    numbers=left, % 行番号を表示(必要な場合)
    numberstyle=\tiny, % 行番号のスタイル
    frame=single, % ソースコードを枠で囲む(必要な場合)
    breaklines=true, % 長い行を自動的に折り返す
    captionpos=b, % キャプションの位置を下にする
}
\renewcommand{\lstlistingname}{ソースコード}

\pagestyle{fancy}
\fancyhf{} % 既存のヘッダーとフッターをクリア
\fancyhead[R]{\thepage}% 右上にページ番号を配置
\setlength{\headheight}{17.0pt}
\addtolength{\topmargin}{-7.0pt}

\begin{document}

\tableofcontents
\clearpage

\section{実験機材}

マイコンボードの通し番号:\ 72

使用したパソコン:\ 実験室のPC(Lenovo Thickpad)

OS:\ Windows

カラーセンサの机からの高さ:\ 1cm程

\section{理論}

\section{実験内容}
\subsection{【課題6】colorsensorの改造}
color\_sensor\_main.cにて,割り込みハンドラをCSPinIntHandlerに設定して割り込みの初期化を行い,main()関数内でカラーセンサの割り込みを許可した.
そして,下敷きの黒色を読み込ませたときに割り込みが発生せず,その他の色を読み込ませたときだけ割り込みが発生するようにした.このとき,事前に色の読み込みデータの値をメモしておき,それらを用いて閾値の設定を行った.
また,得られたカラーセンサの値を利用して,物体の色情報をLCDに表示させた.

\subsection{【課題7】ラーメンタイマーの作成}
rotary\_main.cにて,割り込みハンドラをSW1PinIntHandlerとREPinIntHandlerに設定して割り込みの初期化を行った.
そして,main()関数内でSW1とロータリーエンコーダの割り込みを許可した.REPinIntHandlerで,左へ回すと数値が減少し,右へ回すと数値が増加し,下限と上限がそれぞれ0,600となるようにコードを書いてその結果をLCDへ表示するようにした.
また,SW1PinIntHadlerで割り込みが発生したときにSystick割り込みを許可して数値を1秒ごとに減らしていき,0になったらブザーを鳴らすようにした.そして,その状態でロータリーエンコーダを回すとブザーが停止するようにした.

\subsection{【課題8】色判定器の作成}
color\_sensor\_main.cにて,ロータリーエンコーダの回転に応じて,LCDの1行目の左側に「Red」「Yellow」「Blue」「Green」「None」と表示させ,
2行目左側にはカラーセンサが検知している色を同じように表示させ,1行目右側と2行目右側に各色をカラーセンサが検知した回数を99回まで表示させるように割り込みハンドラを改造した.
また,1行目左側で選択した色と同じ色を検知したら,ブザーから「ラ(O4A)」の音を出し,その他の3色のいずれかであればブザーから「ミ(O4E)」の音を出させるようにコードを書いた.
音は,一定時間が経過すると鳴り止むようにした.SW1を押下することで,色検知カウント回数を0に戻すようにした.

\section{実験結果}

\subsection{【課題6】colorsensorの改造}
以下に,CSPinIntHandlerのソースコードを示す.

\begin{lstlisting}[label=a6CS,caption={【課題6】colorsensorにおけるCSPinIntHandler関数}]
void CSPinIntHandler(void) {
    disableCSPinInt();
    clearCSPinInt();
    clearIntColorSensor();

    setAddressLCD(0, 0);
    writeTextLCD("R:     B:       ", 16);
    setAddressLCD(0, 1);
    writeTextLCD("G:     C:       ", 16);

    uint16_t _red = read16ColorSensor(RDATAL_REG);
    uint16_t _blue = read16ColorSensor(BDATAL_REG);
    uint16_t _green = read16ColorSensor(GDATAL_REG);
    uint16_t _clear = read16ColorSensor(CDATAL_REG);

    uint16_t colors[] = {_red, _blue, _green, _clear};
    int i;
    for(i = 0; i < 4; i++){
        switch(i){
            case 0:
                setAddressLCD(2, 0);
                break;
            case 1:
                setAddressLCD(9, 0);
                break;
            case 2:
                setAddressLCD(2, 1);
                break;
            case 3:
                setAddressLCD(9, 1);
                break;
        }
        writeTextLCD(itoh(colors[i], 4), 4);
    }

    delay_ms(100);

    enableCSPinInt();
}
\end{lstlisting}
この関数では,2行目から4行目にかけて新たなカラーセンサの割り込みを禁止し,カラーセンサ側とマイコンボードのカラーセンサ用のピンにおける割り込み処理をクリアした.
6行目から9行目で一旦RGBCの値を空白として出力し,11行目から14行目で受け取ったRGBCの値を16行目から34行目にかけてLCDに出力させた.36行目で100msだけ処理を遅らせることで連続割り込みを防いだ.
そして,38行目でカラーセンサの割り込みを許可した.

\subsection{【課題7】ラーメンタイマーの作成}
以下にSystickIntHandler,SW1PinIntHandler,REPinIntHandlerのソースコードをそれぞれ示す.
\begin{lstlisting}[label=a7Systick,caption={【課題7】におけるSystickIntHandler関数}]
void SysTickIntHandler(void) {
    if(cnt == 0){
        toneBuzzer(O4C);
        SysTickIntDisable();
        enableREPinInt();
    }

    static uint8_t spaces[] = {' ', ' ', ' ', ' ', ' ', ' ', ' ', ' ', ' ', ' ', ' ', ' ', ' ', ' ', ' ', ' '};

    setAddressLCD(0, 0);
    writeTextLCD(spaces,16);
    setAddressLCD(0, 1);
    writeTextLCD(spaces,16);

    int len = 1, temp = cnt;
    while(temp / 10 > 0){
        len++;
        temp /= 10;
    }
    setAddressLCD(16-len, 0);
    writeTextLCD(itoa(cnt, len), len);
    delay_ms(1000);

    cnt--;
}
\end{lstlisting}

\begin{lstlisting}[label=a7SW1,caption={【課題7】におけるSW1PinIntHandler関数}]
void SW1PinIntHandler(void) {
    disableSW1PinInt();
    clearSW1PinInt();
    disableREPinInt();

    SysTickIntEnable();

    enableSW1PinInt();
}
\end{lstlisting}

\begin{lstlisting}[label=a7RE,caption={【課題7】におけるREPinIntHandler関数}]
void REPinIntHandler(void) {
    disableREPinInt();
    clearREPinInt();

    restBuzzer();

    static uint8_t spaces[] = {' ', ' ', ' ', ' ', ' ', ' ', ' ', ' ', ' ', ' ', ' ', ' ', ' ', ' ', ' ', ' '};

    setAddressLCD(0, 0);
    writeTextLCD(spaces,16);
    setAddressLCD(0, 1);
    writeTextLCD(spaces,16);

    int str_len = 1;
    int32_t d = getDirectionRotaryEncoder();
    cnt += d;
    if(cnt > 600)
        cnt = 600;
    else if(cnt < 0)
        cnt = 0;

    int temp = cnt;
    while(temp / 10 > 0){
        str_len++;
        temp /= 10;
    }

    setAddressLCD(16 - str_len, 0);
    writeTextLCD(itoa(cnt, str_len), str_len);

    enableREPinInt();
}
\end{lstlisting}

ソースコード\ref{a7Systick}は,SW1が押下されてからのカウントタイマーを実装したものである.
具体的には,2行目から6行目でカウントが0になったときにブザー音を鳴らし,Systick割り込みを禁止してロータリーエンコーダの割り込みを許可した.
また,8行目から31行目にかけて現在のカウントをLCDに表示させ,32行目は1秒ごとのタイマーとして扱うために実装した.これらの処理の後,34行目でカウントダウンさせた.

ソースコード\ref{a7SW1}は,SW1が押下されたときの割り込みの呼び出しや停止を実装したものである.
具体的には,

\subsection{【課題8】色判定器の作成}

\section{考察}
\paragraph{【考察1】}

\paragraph{【考察2】}

\paragraph{【考察3】}

\paragraph{【考察4】}

\section{問題回答}
提出期限超過のため割愛

\end{document}