\documentclass{jlreq}

\usepackage{amsmath}
\usepackage{bm}
\usepackage{fancyhdr}
\usepackage{float}
\usepackage{graphicx}
\usepackage{physics}
\usepackage{siunitx}

\numberwithin{equation}{section}

\pagestyle{fancy}
\fancyhf{}
\fancyhead[R]{\thepage}

\begin{document}

\tableofcontents
\clearpage

\section{目的}
人間の認知処理がどのような場合に速く,または遅くなるのかを調べて,インターフェース設計の
基礎となるアイコンやキーの対応について考える.

% \section{実験機材}
% 使用した機材は,Dell\ Inspiron\ 15\ 3535である.OSはWindows11\ Homeであり,用いたR言語はR\ version\ 4.3.2である.

\section{実験方法}
認知処理速度や正答率などを計測するための3つの認知課題をカウンターバランスを意識して行った.

\subsection{Task1}
\subsection{Task2}
\subsection{Task3}

得られたログファイルを表計算ソフトで管理し,各課題について「全試行」,「正答した試行のみ」を対象にした一致条件と不一致条件
の結果を整理した.また,2班でデータの共有を行った.

\section{結果}

\section{考察}

\begin{thebibliography}{9}
  \item 西崎友規子.プロジェクト実習Ⅰ ヒューマンインターフェース 実験テキスト.京都工芸繊維大学,2024年
\end{thebibliography}

\end{document}